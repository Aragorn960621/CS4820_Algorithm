\documentclass[11pt]{article}
%\usepackage{fullpage}
\usepackage{epic}
\usepackage{eepic}
\usepackage{paralist}
\usepackage{algorithm,algorithmic}
\usepackage{amsfonts,amsmath}


%%%%%%%%%%%%%%%%%%%%%%%%%%%%%%%%%%%%%%%%%%%%%%%%%%%%%%%%%%%%%%%%
% This is FULLPAGE.STY by H.Partl, Version 2 as of 15 Dec 1988.
% Document Style Option to fill the paper just like Plain TeX.

\typeout{Style Option FULLPAGE Version 2 as of 15 Dec 1988}

\topmargin 0pt
\advance \topmargin by -\headheight
\advance \topmargin by -\headsep

\textheight 8.9in

\oddsidemargin 0pt
\evensidemargin \oddsidemargin
\marginparwidth 0.5in

\textwidth 6.5in
%%%%%%%%%%%%%%%%%%%%%%%%%%%%%%%%%%%%%%%%%%%%%%%%%%%%%%%%%%%%%%%%

\pagestyle{empty}
\setlength{\oddsidemargin}{0in}
\setlength{\topmargin}{-0.8in}
\setlength{\textwidth}{6.8in}
\setlength{\textheight}{9.5in}


\def\ind{\hspace*{0.3in}}
\def\gap{0.1in}
\def\bigap{0.25in}
\newcommand{\Xomit}[1]{}


\begin{document}

\setlength{\parindent}{0in}
\addtolength{\parskip}{0.1cm}
\setlength{\fboxrule}{.5mm}\setlength{\fboxsep}{1.2mm}
\newlength{\boxlength}\setlength{\boxlength}{\textwidth}
\addtolength{\boxlength}{-4mm}
\begin{center}\framebox{\parbox{\boxlength}{{\bf
CS 4820, Spring 2018 \hfill Homework 11, Problem 1}\\
% TODO: fill in your own name, netID, and collaborators
Name: \\
NetID: \\
Collaborators:
}}
\end{center}
\vspace{5mm}

{ \bf (1)} {\em (10 points)}
For each of the following optimization problems, present an 
integer program whose optimum value matches the optimum
value of the given problem.  The combined number of variables 
and constraints in your integer program should be polynomial
in the size of the given instance of the optimization problem.
\renewcommand{\theenumi}{(\roman{enumi})}
\begin{enumerate}
\item {\sc Set Cover}.  Given a universal set $\mathcal{U}$
and a collection of subsets $S_1,S_2,\ldots,S_m \subseteq \mathcal{U}$,
what is the minimum size of a subcollection
$\{S_{i_1},S_{i_2},\ldots,S_{i_k}\}$ whose union is $\mathcal{U}$?
\item  {\sc Independent Set}.  Given a graph $G=(V,E)$, 
find an independent set of maximum cardinality.
\item  {\sc Max-3SAT}.  Given a set of Boolean variables
$x_1,x_2\ldots,x_n$ and a set of clauses $C_1,C_2,\ldots,C_m$ each 
consisting of a disjunction of 3 literals from the set
$\{x_1,\bar{x}_1,\ldots,x_n,\bar{x}_n\}$, 
what is the  maximum number of clauses that can be 
satisfied by a truth assignment?
\item  {\sc Max-Cut}.  Given an undirected graph $G=(V,E)$,
what is the maximum size of a cut?
(A cut $(A,B)$ is any partition of the vertex set $V$ into two 
nonempty subsets.  The size of a cut is equal to the number
of edges with one
endpoint on each side of the partition.)
\end{enumerate}
It is not necessary to prove that your answer is valid.
However, you should explain the interpretation of 
your notation well enough 
that we completely understand the structure of your
integer program.

{\bf Example:}  In the weighted vertex cover problem, one
is given a graph $G=(V,E)$ and a non-negative weight $w_v$
for every vertex $v \in V$.  One is asked to find the
minimum total weight of a vertex cover.\\[1em]
{\sc Answer:}  The equivalent
integer program is:
\begin{eqnarray*}
\text{min} & & \sum_{v \in V} w_v x_v \\
\text{s.t.}
& & x_u + x_v \geq 1 \;\; \mbox{for all edges $e = (u,v)$} \\
& & x_v \in \{0,1\} \;\;\; \mbox{for all vertices $v$}
\end{eqnarray*}
{\sc Interpretation:} Decision variable $x_u$ equals $1$
if vertex $u$ is included in the vertex cover, $0$ otherwise.


\vskip \bigap

%% Your solution goes here.

\end{document}
