\documentclass[11pt]{article}
%\usepackage{fullpage}
\usepackage{epic}
\usepackage{eepic}
\usepackage{paralist}
\usepackage{graphicx}
\usepackage{algorithm,algorithmic}
\usepackage{tikz}
\usepackage{xcolor,colortbl}
\usepackage{wrapfig}


%%%%%%%%%%%%%%%%%%%%%%%%%%%%%%%%%%%%%%%%%%%%%%%%%%%%%%%%%%%%%%%%
% This is FULLPAGE.STY by H.Partl, Version 2 as of 15 Dec 1988.
% Document Style Option to fill the paper just like Plain TeX.

\typeout{Style Option FULLPAGE Version 2 as of 15 Dec 1988}

\topmargin 0pt
\advance \topmargin by -\headheight
\advance \topmargin by -\headsep

\textheight 8.9in

\oddsidemargin 0pt
\evensidemargin \oddsidemargin
\marginparwidth 0.5in

\textwidth 6.5in
%%%%%%%%%%%%%%%%%%%%%%%%%%%%%%%%%%%%%%%%%%%%%%%%%%%%%%%%%%%%%%%%

\pagestyle{empty}
\setlength{\oddsidemargin}{0in}
\setlength{\topmargin}{-0.8in}
\setlength{\textwidth}{6.8in}
\setlength{\textheight}{9.5in}


\def\ind{\hspace*{0.3in}}
\def\gap{0.1in}
\def\bigap{0.25in}
\newcommand{\Xomit}[1]{}


\begin{document}

\setlength{\parindent}{0in}
\addtolength{\parskip}{0.1cm}
\setlength{\fboxrule}{.5mm}\setlength{\fboxsep}{1.2mm}
\newlength{\boxlength}\setlength{\boxlength}{\textwidth}
\addtolength{\boxlength}{-4mm}
\begin{center}\framebox{\parbox{\boxlength}{{\bf
CS 4820, Spring 2018 \hfill Homework 9, Problem 1}\\
% TODO: fill in your own name, netID, and collaborators
Name: \\
NetID: \\
Collaborators:
}}
\end{center}
\vspace{5mm}

{\bf (2)} {\em (10 points)}
Let us call a Turing machine $M$ ``speedy'' if it has the following
property: for every input string $x$, the computation of $M$ on 
input $x$ terminates, and it does so after at most $|x| + 100$ steps. 

Design an algorithm to decide whether a Turing machine is speedy,
given a description of the Turing machine. In more detail, the
input to your algorithm consists {\em only} of the description
of a machine $M$ (in particular the input to your algorithm does
not specify any particular input string for $M$) and its output
should be ``yes'' if $M$ is speedy and ``no'' if $M$ is not speedy.

In your solution you should describe the algorithm --- an ordinary
algorithm description is fine, no need to explain how to implement
the algorithm as a Turing machine. Prove that your algorithm
always terminates, and prove that it always outputs the correct
answer. However, {\em you do not need to analyze your algorithm's
running time, other than proving that it always terminates.}


\vskip \bigap

%% Your solution goes here.

\end{document}
