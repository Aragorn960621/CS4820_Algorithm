\documentclass[11pt]{article}
%\usepackage{fullpage}
\usepackage{epic}
\usepackage{eepic}
\usepackage{paralist}
\usepackage{graphicx}
\usepackage{algorithm,algorithmic}
\usepackage{tikz}
\usepackage{xcolor,colortbl}
\usepackage{wrapfig}


%%%%%%%%%%%%%%%%%%%%%%%%%%%%%%%%%%%%%%%%%%%%%%%%%%%%%%%%%%%%%%%%
% This is FULLPAGE.STY by H.Partl, Version 2 as of 15 Dec 1988.
% Document Style Option to fill the paper just like Plain TeX.

\typeout{Style Option FULLPAGE Version 2 as of 15 Dec 1988}

\topmargin 0pt
\advance \topmargin by -\headheight
\advance \topmargin by -\headsep

\textheight 8.9in

\oddsidemargin 0pt
\evensidemargin \oddsidemargin
\marginparwidth 0.5in

\textwidth 6.5in
%%%%%%%%%%%%%%%%%%%%%%%%%%%%%%%%%%%%%%%%%%%%%%%%%%%%%%%%%%%%%%%%

\pagestyle{empty}
\setlength{\oddsidemargin}{0in}
\setlength{\topmargin}{-0.8in}
\setlength{\textwidth}{6.8in}
\setlength{\textheight}{9.5in}


\def\ind{\hspace*{0.3in}}
\def\gap{0.1in}
\def\bigap{0.25in}
\newcommand{\Xomit}[1]{}


\begin{document}

\setlength{\parindent}{0in}
\addtolength{\parskip}{0.1cm}
\setlength{\fboxrule}{.5mm}\setlength{\fboxsep}{1.2mm}
\newlength{\boxlength}\setlength{\boxlength}{\textwidth}
\addtolength{\boxlength}{-4mm}
\begin{center}\framebox{\parbox{\boxlength}{{\bf
CS 4820, Spring 2018 \hfill Homework 9, Problem 1}\\
% TODO: fill in your own name, netID, and collaborators
Name: \\
NetID: \\
Collaborators:
}}
\end{center}
\vspace{5mm}

{ \bf (1)} {\em (10 points)}
Design a single-tape Turing machine to evaluate the ``less than'' 
relation on two natural numbers represented in binary.
The input to the Turing machine is represented as a 
string over the alphabet $\{0,1,>,?\}$. The input string
is always in the format $a < b ?$, where each of $a$ and 
$b$ is a string of one or more binary digits, beginning
with the digit 1. Your
Turing machine should terminate in the ``yes'' state
if the number represented by $a$ in binary is strictly
less than the number represented by $b$ in binary,
and it should terminate in the ``no'' state if the number
represented by $a$ in binary is greater than or equal 
to the number represented by $b$ in binary. If the
input violates the format requirements, any behavior
is acceptable as long as your algorithm terminates.

{\bf Example:} if the input is $11<100?$ the answer is ``yes''.
If the input is $0011<100?$ your algorithm should terminate,
but it is fine to terminate in any of the ``yes'', ``no'', or
``halt'' states because the first binary string does not
begin with the digit 1.

Your answer must include a description, in English (with
accompanying notation as needed), of the alphabet, state set, and 
transition rule of your Turing machine, and a few sentences
explaining how the algorithm operates and how the specified
states and transitions implement those operations. You may choose to
include a representation of the transition rule in tabular 
form (similar to those represented in Section 2 of the lecture notes) if
you wish, but the tabular representation of the Turing machine
is {\em optional} whereas the human-interpretable explanation
is {\em mandatory}.

Your solution should include an analysis of the worst-case 
running time, as a function of the length of the input string. 
{\em You do not need to write a proof of 
correctness.}


\vskip \bigap

%% Your solution goes here.

\end{document}
