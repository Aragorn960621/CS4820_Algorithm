\documentclass[11pt]{article}
%\usepackage{fullpage}
\usepackage{epic}
\usepackage{eepic}
\usepackage{paralist}
\usepackage{graphicx}
\usepackage{algorithm,algorithmic}
\usepackage{tikz}
\usepackage{xcolor,colortbl}
\usepackage{wrapfig}


%%%%%%%%%%%%%%%%%%%%%%%%%%%%%%%%%%%%%%%%%%%%%%%%%%%%%%%%%%%%%%%%
% This is FULLPAGE.STY by H.Partl, Version 2 as of 15 Dec 1988.
% Document Style Option to fill the paper just like Plain TeX.

\typeout{Style Option FULLPAGE Version 2 as of 15 Dec 1988}

\topmargin 0pt
\advance \topmargin by -\headheight
\advance \topmargin by -\headsep

\textheight 8.9in

\oddsidemargin 0pt
\evensidemargin \oddsidemargin
\marginparwidth 0.5in

\textwidth 6.5in
%%%%%%%%%%%%%%%%%%%%%%%%%%%%%%%%%%%%%%%%%%%%%%%%%%%%%%%%%%%%%%%%

\pagestyle{empty}
\setlength{\oddsidemargin}{0in}
\setlength{\topmargin}{-0.8in}
\setlength{\textwidth}{6.8in}
\setlength{\textheight}{9.5in}


\def\ind{\hspace*{0.3in}}
\def\gap{0.1in}
\def\bigap{0.25in}
\newcommand{\Xomit}[1]{}


\begin{document}

\setlength{\parindent}{0in}
\addtolength{\parskip}{0.1cm}
\setlength{\fboxrule}{.5mm}\setlength{\fboxsep}{1.2mm}
\newlength{\boxlength}\setlength{\boxlength}{\textwidth}
\addtolength{\boxlength}{-4mm}
\begin{center}\framebox{\parbox{\boxlength}{{\bf
CS 4820, Spring 2018 \hfill Homework 4, Problem 1}\\
% TODO: fill in your own name, netID, and collaborators
Name: \\
NetID: \\
Collaborators:
}}
\end{center}
\vspace{5mm}

{\bf (1)} {\em (Note: This problem is Exercise 5.3 in Kleinberg \& Tardos,
with an extra ``part b'' appended.)}\\
Suppose you're consulting for a bank that's concerned about 
fraud detection, and they come to you with the following
problem. They have a collection of $n$ bank cards that they've
confiscated, suspecting them of being used in fraud. Each bank 
card is a small plastic object, containing a magnetic stripe
with some encrypted data, and it corresponds to a unique 
account in the bank. Each account can have many bank cards
corresponding to it, and we'll say that two bank cards are
{\em equivalent} if they correspond to the same account.

It's very difficult to read the account number off a bank card
directly, but the bank has a high-tech ``equivalence tester'' that
takes two bank cards and, after performing some computations,
determines whether they are equivalent.

Their question is the following: among the collection of 
$n$ cards, is there a set of more than $n/2$ of them that are
all equivalent to one another? Assume that the only feasible 
operations you can do with the cards are to pick two of them
and plug them into the equivalence tester. 

\vskip \gap
{\bf (a)} {\em (5 points)}
Show how to decide the answer to their question with 
only $O(n \log n)$ invocations of the equivalence tester.

\vskip \gap
{\bf (b)} {\em (5 points)}
Now modify the question so that you must decide whether there
is a set of more than $n/4$ of the cards that are all equivalent
to one another. Show how to decide the answer to this question
with only $O(n \log n)$ invocations of the equivalence
tester. 

{\bf \itshape{If you are confident that you have a correct solution to
part (b) of this problem, you can skip part (a) and 
your score on the problem will be computed by doubling
your score for part (b).}}


\vskip \bigap

%% Your solution goes here.

\end{document}
