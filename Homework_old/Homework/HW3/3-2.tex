\documentclass[12pt]{article}
%\usepackage{fullpage}
\usepackage{epic}
\usepackage{eepic}
\usepackage{paralist}
\usepackage{graphicx}
\usepackage{algorithm,algorithmic}
\usepackage{tikz}
\usepackage{xcolor,colortbl}
\usepackage{wrapfig}


%%%%%%%%%%%%%%%%%%%%%%%%%%%%%%%%%%%%%%%%%%%%%%%%%%%%%%%%%%%%%%%%
% This is FULLPAGE.STY by H.Partl, Version 2 as of 15 Dec 1988.
% Document Style Option to fill the paper just like Plain TeX.

\typeout{Style Option FULLPAGE Version 2 as of 15 Dec 1988}

\topmargin 0pt
\advance \topmargin by -\headheight
\advance \topmargin by -\headsep

\textheight 8.9in

\oddsidemargin 0pt
\evensidemargin \oddsidemargin
\marginparwidth 0.5in

\textwidth 6.5in
%%%%%%%%%%%%%%%%%%%%%%%%%%%%%%%%%%%%%%%%%%%%%%%%%%%%%%%%%%%%%%%%

\pagestyle{empty}
\setlength{\oddsidemargin}{0in}
\setlength{\topmargin}{-0.8in}
\setlength{\textwidth}{6.8in}
\setlength{\textheight}{9.5in}


\def\ind{\hspace*{0.3in}}
\def\gap{0.1in}
\def\bigap{0.25in}
\newcommand{\Xomit}[1]{}


\begin{document}

\setlength{\parindent}{0in}
\addtolength{\parskip}{0.1cm}
\setlength{\fboxrule}{.5mm}\setlength{\fboxsep}{1.2mm}
\newlength{\boxlength}\setlength{\boxlength}{\textwidth}
\addtolength{\boxlength}{-4mm}
\begin{center}\framebox{\parbox{\boxlength}{{\bf
CS 4820, Spring 2018 \hfill Homework 3, Problem 2}\\
% TODO: fill in your own name, netID, and collaborators
Name: \\
NetID: \\
Collaborators:
}}
\end{center}
\vspace{5mm}

{\bf (2)} 
In class we've been talking about applications of
dynamic programming to optimization. There are also
many applications of dynamic programming to counting,
and to calculating probabilities. This exercise explores
one such application. Recall that an instance of
the {\em interval scheduling} problem consists of
$n$ intervals $I_1,I_2,\ldots,I_n$, where each interval
$I_k$ (for $k=1,\ldots,n$) is a closed interval
$[s_k,f_k]$ with start time $s_k$ and finish time
$f_k > s_k$. A set of intervals is {\em non-conflicting}
if no two of its elments overlap. 

In this exercise
we assume we are given an instance of the interval
scheduling problem such that the numbers 
$s_1,s_2,\ldots,s_n,f_1,f_2,\ldots,f_n$ 
are all distinct, and such that the intervals
are ordered by increasing finish time:
$f_1 < f_2 < \cdots < f_n$.

\vskip \gap
{\bf (2a)} {\em (7 points)} \\
Design an algorithm to count how
many subsets of $\{I_1,I_2,\ldots,I_n\}$ are
non-conflicting. Remember that the empty set
and one-element sets are always non-conflicting.

\vskip \gap
{\bf (2b)} {\em (3 points)} \\
Let $\Omega$ denote the collection of all
non-conflicting subsets of $\{I_1,\ldots,I_n\}$.
Given the list of intervals $I_1,\ldots,I_n$, and
an index $k$ in the range $1 \le k \le n$, 
design an algorithm to compute the probability that
a uniformly random element of $\Omega$ contains
interval $I_k$.

In your solution to (2b), you may omit the running
time analysis. The algorithm you design must
still have running time bounded by a polynomial
function of $n$, but you don't need to include the
analysis of running time in your write-up. You
are also free to use the algorithm from part (2a)
as a subroutine in part (2b), even if you didn't
succeed in solving (2a).


\vskip \bigap

%% Your solution goes here.

\end{document}
